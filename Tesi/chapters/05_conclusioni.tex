\chapter{Conclusioni e prospettive future} \label{chapter:end}
Il tirocinio che ha dato luogo alla creazione del prototipo oggetto di questa relazione è stato molto utile dal punto di vista professionale. L'accesso alla conoscenza di persone con molta esperienza in questo settore è sicuramente importante per la crescita personale, soprattutto dal punto di vista lavorativo. Durante il periodo di stage posso affermare di essere stato supportato in modo eccellente e istruito mediante corsi di formazione inerenti alle tecnologie utilizzate per questo progetto. Al termine del tirocinio, ho avuto e colto l'opportunità di continuare la mia esperienza in quest'azienda.\\

In conclusione, il prototipo progettato e implementato fornisce una soluzione per l'informatizzazione di un processo che attualmente risulta laborioso, quindi lento, e soggetto ad eventuali errori, che rallenterebbero ulteriormente il procedimento di rendicontazione dell'attività lavorativa di un dipendente.\\ 

\noindent
Gli eventuali sviluppi futuri potrebbero riguardare la sicurezza, in particolare migliorare la gestione dell'autenticazione, per esempio criptando la password dell'utente con un opportuno algoritmo di hashing, diversamente da quanto viene fatto attualmente. Inoltre si potrebbe migliorare la gestione dei permessi di lavoro e malattia, introducendo la richiesta di giustificativi e certificati medici. Infine, sarebbe utile integrare una funzionalità che permette al dipendente di accedere alla sua busta paga, quindi visualizzarla e scaricarla localmente.

