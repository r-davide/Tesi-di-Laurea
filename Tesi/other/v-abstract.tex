\newpage
\cleardoublepage
\begingroup
\let\clearpage\endgroup
\null\vspace{\stretch{1}}
\chapter*{\centering Abstract}

L'informatizzazione dei processi aziendali gioca un ruolo assai rilevante e ha certamente impatto sull'efficienza generale di un'azienda, nonché sui costi che questa deve sostenere. Questo progetto, frutto di un periodo di stage curricolare, ha come obiettivo quello di informatizzare la rendicontazione dell'attività lavorativa di un dipendente. Con tale elaborato si vogliono, quindi, presentare quelle che sono state le varie fasi coinvolte durante la progettazione e lo sviluppo di tale software, motivando le varie decisioni intraprese. Vengono inoltre introdotti concetti teorici, riguardanti le tecnologie utilizzate, indispensabili per la comprensione del funzionamento generale dell'applicazione e nello specifico per capire l'interazione tra i vari elementi che la compongono dal punto di vista tecnico. Tali concetti sono altresì utili per capire le diverse scelte di implementazione concepite.

\vspace{\stretch{2}} \null


%Le reti Bayesiane sono modelli grafico-probabilistici utilizzati per effettuare inferenze,date delle evidenze. Questo lavoro di stage studia come sia possibile lagenerazionediquesti modellia partire da ontologie, vale a dire rappresentazioni dettagliate di domini diinteresse.Per cominciare, viene presentato il problema della traduzione in rete Bayesiana diun’ontologia. Vengono inoltre introdotti i concetti teorici fondamentali per la comprensionedel lavoro svolto.L’elaborato procede con un’analisi approfondita della letteratura riguardo al tematrattato, dalla quale emergono sfide ambiziose da affrontare e molti problemi complessiche richiedono ulteriori sforzi di ricerca. Sfide e problemi sono presentati tramite unalettura critica delle pubblicazioni scientifiche relative all’attuale stato dell’arte.Viene inoltre proposto un approccio che utilizza strumenti pre-esistenti e li coordinain una pipeline semi-automatica per trasformare un’ontologia in una rete Bayesiana, sullaquale condurre processi inferenziali. Alcuni esempi d’applicazione di questa metodologiavengono presentati e discussi al fine di dimostrarne l’effettiva validità e di mettere in lucepregi e difetti di questo approccio
